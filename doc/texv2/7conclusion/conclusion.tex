% !TeX root = ../DistributedConsensus.tex
% !TeX spellcheck = en_GB
\chapter{Conclusion}
This report has examined how it is possible to find a global order of execution of a distributed DCR graph. Several algorithms are used in order to accomplish this order of execution. By storing executions as actions performed by events as a totally ordered history on these individual events and using Lamport's logical clocks as timestamps in order to establish happens-before relations, it is possible to merge local histories together. A combined history can be simplified to represent only the order of executions by collapsing actions of executions together to form single entities with the same happens-before relations, and finally using transitive reduction to find the minimum equivalent graph of the order of execution.

The observability and identifiability of malicious behaviour of processes in the DCR graph hosting events depends on both the type of cheating and the structure of the DCR graph. By applying validations on individual histories of events, pair validations between histories of two events and simulating the order of execution in the DCR graph, it is possible to in most cases observe malicious behaviour, and as a minimum state which processes \textit{could} be the reason behind the observed malicious behaviour. If the amount of interconnectivity of the DCR graph is sparse, and malicious nodes are interconnected, then it is difficult to observe if some kind of cheating has happened at all.

To reach distributed consensus on the order of execution, each event can examine the resulting order of execution and confirm that the order of the individual executions of the events are preserved in order in the global order of execution.

\todo[inline]{det er godt nok en lidt lang opsummering men jeg synes egentlig det alligevel er en okay mængde i en konklusion. Det er jo meningen at man skal kunne læse introen og den og have en nogenlunde ide}

\newpar We have tried multiple approaches to solving the problem over the course of this project. Many have been resulted in reaching a dead end when discovering problems not easily solved or without putting up requirements that would make the possible use case of the algorithms too strict\todo{Fisk: Andet ord?}. The fact that at most two events have to agree on information and that no event knows the global order of execution makes it impossible to use election algorithms to handle malicious behavior of processes. Furthermore, gathering information in a peer-to-peer fashion were shown as giving less information than desired and therefore made observing malicious processes infeasible.

\newpar Overall, algorithms have been described that solve the problem in some \todo{Fisk: Før stod der "most".} cases. Multiple areas could be extended upon in future projects, for example how malicious nodes are handled if they are identified, with which structures of DCR graphs one could prevent malicious behaviour described in this project completely, and finally in what cases it is impossible to find and order of execution at all. Furthermore a peer-to-peer based gathering algorithm that would be able to handle malicious events could be researched.


\todo[inline]{man kunne undersøge hvad man kan gøre for at redde et workflow hvor der er nogen der har snydt}