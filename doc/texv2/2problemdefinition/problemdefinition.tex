% !TeX root = ../DistributedConsensus.tex
% !TeX spellcheck = en_GB
\chapter{Problem Definition} % Husk overgang med amerikansk lærebogsafsnit
It is desired to find the order of execution in a DCR graph. Consensus should therefore be reached between nodes in the graph about the order of execution. Finding this order raises several problems.

Finding a history or log of a given order of execution can be difficult in distributed DCR graphs, due to the fact that no single node has the overview of the entire workflow in the graph.
Logs can be split among several nodes, timestamps of logs do not necessarily correspond among nodes and nodes can emit erroneous logs. These challenges must be overcome in order to reach consensus on the order of execution.

\newpar
Given a distributed implementation of DCR Graphs, the purpose is to extract an order of execution in the instance of a workflow. Furthermore the computers running the events should agree upon this order of execution.

\newpar
In \autoref{chap:theory} relevant theory needed in the rest of the report is described. This includes theory of DCR Graphs as well as certain aspects of Distributed Systems. In \autoref{chap:representing-a-history} we describe a simplified problem, where the DCR Graph contains a single event. It is discussed how to represent \texttt{Action}s and \texttt{History}. In \autoref{chap:connecting-histories} the problem is extended by adding neighbouring events to the starting event in the workflow. In \autoref{chap:gathering-distributed-history} it is discussed how we can gather histories of all events in a connected DCR Graph, and which requirements it poses on the workflow. In \autoref{chap:consensusindcr} it is elaborated how to reach consensus on a given resulting history, as well as how to cope with events emitting erroneous data.