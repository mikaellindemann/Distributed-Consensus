% !TeX root = ../DistributedConsensus.tex
% !TeX spellcheck = en_GB
\chapter{Introduction}
	Dynamic Condition Response Graphs (DCR graphs, see \cite{hildebrandt2011declarative}) are developed principally by Thomas Hildebrandt at the IT University of Copenhagen, in collaboration with ResultMaker\footnote{\url{http://www.resultmaker.com}} and later Exformatics A/S\footnote{\url{http://www.exformatics.dk}}. DCR graphs are used to model workflows which represents work processes.
	
	\newpar DCR graphs are used by companies to ensure that business procedures are executed according to business rules. At a given time a business might want to see in which order events in a given DCR graph have executed, a so called \textit{order of execution}. The order of execution can contain interesting information for the business, as it reveals which events have been executed to lead to the current state of the workflow. Furthermore, if a DCR graph is used to represent a case of a customer, and if multiple case workers are working on that case, and do not have a total overview of it, an order of execution can provide that overview. 
	
	Since DCR graphs allow infinite behaviour, the workflow can be in the same state multiple times. It is possible to extract the previous identical states and the events that have occurred in order to get to that state from an order of execution. Finally, for DCR graphs shared between two or more companies, one of the participating companies might want to check retroactively whether the other companies have followed the rules of the workflow.
	
	\newpar The events of a DCR graph can be distributed over a network allowing better scalability, reliability as well as allowing multiple companies to work together, by hosting and handling different events themselves.
	
	\newpar In a distributed DCR graph finding an order of execution can be difficult, because no single event has an overview over the entire workflow. Information about what has happened can be split among several events, the clocks of events are not necessarily synchronised, and events may emit erroneous information. Even though there are challenges, DCR graphs provide information that can be used to relate executions with one another, which is especially helpful when logs of two or more events need to be combined into an order of execution. This order of execution should be able to filter information from and inform about malicious events in the workflow.

	\newpar By researching and applying distributed system theories and algorithms, as well as the rules of DCR graphs, we will try to solve the problem of generating and reaching consensus of histories of DCR graphs.